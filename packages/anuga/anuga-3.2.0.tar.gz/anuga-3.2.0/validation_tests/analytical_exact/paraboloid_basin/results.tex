\section{Oscillations on a paraboloid basin}
This test simulates water oscillations on a paraboloid basin. The analytical solution was derived by Thacker~\cite{Thacker1981}, and is periodic. At any instant in time, the free surface elevation is paraboloid, and the velocity is linear. The scenario includes regular wetting and drying, as the flow oscillates up and down in the basin. As well as testing the ability of the code to do wetting and drying, it will highlight any numerical energy loss or gain, and manifest as an increase or decrease in the magnitude of the flow oscillations over long time periods (compared with the analytical solution). This test was also implemented by Yoon and Cho~\cite{YC2001} to investigate the performance of their numerical method.

Consider the topography in two dimensions
\begin{equation}
z(x,y) = -D_0\left[1 -\left(\frac{r}{L}\right)^2\right]
\end{equation}
where $r=\sqrt{x^2 + y^2}$. Here $D_0$ is the largest depth when water is still and $L$ is the distance between the centre of water surface and the shore when water is still.
The analytical solution is 
\begin{equation}
u(x,y,t) = \frac{\omega r A \sin{(\omega t)}}{ 2 \left[1 -A \cos(\omega t)\right] },
\end{equation}
\begin{equation}
w(x,y,t) = D_0 \left[\frac{\sqrt(1-A^2)}{1-A\cos(\omega t)}  -1 
-\left( \frac{r}{L}\right)^2 \frac{1-A^2}{[(1-A\cos(\omega t))^2]-1} \right].
\end{equation}
Here $\omega=\frac{2\sqrt{2 g D_0}}{L}$ and $A = \frac{L^4 - R_0^4}{L^4 + R_0^4}$ and $R_0$ is the horizontal distance between the centre of water surface and the shore at the initial condition.
The initial condition is set by taking $t=0$ in the analytical solution.

\subsection{Results}
For our test, we consider $D_0=1000$, $L=2500$, and $R_0=2000$.
After running the simulation for some time, we have Figures~\ref{fig:cs_stage}--\ref{fig:cs_xvel} showing the stage, $x$-momentum, and $x$-velocity respectively. There should be a good agreement between numerical and analytical solutions, although wet-dry artefacts may appear in the 'nearly-dry' areas, where the numerical method can cause the water to drain too slowly (similar to that reported in \cite{KESSERWANIA14} using a finite volume scheme with similarities to discontinuous elevation algorithms in \anuga{}).

As time goes on, some small deviations may appear. These are shown in Figures~\ref{fig:w_centre}--\ref{fig:u_centre}, which illustrate the stage, $x$-momentum, and $x$-velocity at the centroid of the domain.


\begin{figure}
\begin{center}
\includegraphics[width=0.9\textwidth]{cross_section_stage.png}
\caption{Stage on a cross section of the basin at time $t=50$\,.}
\label{fig:cs_stage}
\end{center}
\end{figure}

\begin{figure}
\begin{center}
\includegraphics[width=0.9\textwidth]{cross_section_xmom.png}
\caption{Xmomentum on a cross section of the basin at time $t=50$\,.}
\label{fig:cs_xmom}
\end{center}
\end{figure}

\begin{figure}
\begin{center}
\includegraphics[width=0.9\textwidth]{cross_section_xvel.png}
\caption{Xvelocity on a cross section of the basin at time $t=50$\,.}
\label{fig:cs_xvel}
\end{center}
\end{figure}




\begin{figure}
\begin{center}
\includegraphics[width=0.9\textwidth]{Stage_origin.png}
\caption{Stage over time in the centre of the paraboloid basin.}
\label{fig:w_centre}
\end{center}
\end{figure}

\begin{figure}
\begin{center}
\includegraphics[width=0.9\textwidth]{Xmom_origin.png}
\caption{Xmomentum over time in the centre of the paraboloid basin.}
\label{fig:p_centre}
\end{center}
\end{figure}

\begin{figure}
\begin{center}
\includegraphics[width=0.9\textwidth]{Xvel_origin.png}
\caption{Xvelocity over time in the centre of the paraboloid basin.}
\label{fig:u_centre}
\end{center}
\end{figure}


\endinput
