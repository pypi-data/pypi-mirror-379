
\documentclass[12pt]{article}
%%%%%%%%%%%%%%%%%%%%%%%%%%%%%%%%%%%%%%%%%%%%%%%%%%%%%%%%%%%%%%%%%%%%%%%%%%%%%%%%%%%%%%%%%%%%%%%%%%%%%%%%%%%%%%%%%%%%%%%%%%%%%%%%%%%%%%%%%%%%%%%%%%%%%%%%%%%%%%%%%%%%%%%%%%%%%%%%%%%%%%%%%%%%%%%%%%%%%%%%%%%%%%%%%%%%%%%%%%%%%%%%%%%%%%%%%%%%%%%%%%%%%%%%%%%%
\usepackage{amsfonts}
\usepackage{amssymb}
\usepackage{sw20elba}

%TCIDATA{OutputFilter=LATEX.DLL}
%TCIDATA{Version=5.50.0.2890}
%TCIDATA{<META NAME="SaveForMode" CONTENT="1">}
%TCIDATA{BibliographyScheme=Manual}
%TCIDATA{Created=Thursday, June 27, 2013 17:23:01}
%TCIDATA{LastRevised=Sunday, June 30, 2013 11:35:54}
%TCIDATA{<META NAME="GraphicsSave" CONTENT="32">}
%TCIDATA{<META NAME="DocumentShell" CONTENT="Articles\SW\mrvl">}
%TCIDATA{CSTFile=LaTeX article (bright).cst}
%TCIDATA{ComputeDefs=
%$A=\left( 
%\begin{array}{cc}
%t & x \\ 
%y & z%
%\end{array}%
%\right) $
%$P=\left( 
%\begin{array}{ll}
%-1 & 0 \\ 
%0 & 1%
%\end{array}%
%\right) $
%}


\newtheorem{theorem}{Theorem}
\newtheorem{axiom}[theorem]{Axiom}
\newtheorem{claim}[theorem]{Claim}
\newtheorem{conjecture}[theorem]{Conjecture}
\newtheorem{corollary}[theorem]{Corollary}
\newtheorem{definition}[theorem]{Definition}
\newtheorem{example}[theorem]{Example}
\newtheorem{exercise}[theorem]{Exercise}
\newtheorem{lemma}[theorem]{Lemma}
\newtheorem{notation}[theorem]{Notation}
\newtheorem{problem}[theorem]{Problem}
\newtheorem{proposition}[theorem]{Proposition}
\newtheorem{remark}[theorem]{Remark}
\newtheorem{solution}[theorem]{Solution}
\newtheorem{summary}[theorem]{Summary}
\newenvironment{proof}[1][Proof]{\noindent\textbf{#1.} }{{\hfill $\Box$ \\}}
\input{tcilatex}
\addtolength{\textheight}{30pt}

\begin{document}

\title{Algebra 5.45}
\author{Michael Vaughan-Lee}
\date{June 2013}
\maketitle

Algebra 5.45 has $p$ immediate descendants of order $p^{6}$. These $p$
descendants are given by a two parameter family of Lie rings, named 6.427.

The two parameters are $x,y$, and the pair $(x,y)$ gives the same algebra as 
$(z,t)$ if and only if $y^{2}-\omega x^{2}=t^{2}-\omega z^{2}\func{mod}p$.
(Here, as elsewhere, $\omega $ is a primitive element modulo $p$.) We get
the $\frac{p+1}{2}$ distinct squares modulo $p$ with parameters $(x,0)$ with 
$0\leq x\leq \frac{p-1}{2}$. To obtain the non-squares, find $a$ such that $%
a^{2}-\omega $ is not a square modulo $p$, and take parameters $(ay,y)$ for $%
0<y\leq \frac{p-1}{2}$. In the case $p=1\func{mod}4$, $a=0$ will do. I don't
think the search for $a$ is linear in $p$ for $p=3\func{mod}4$, but since $%
a^{2}-\omega $ is not a square modulo $p$ for half of the possible values of 
$a$, you would have to be unlucky not to find a suitable $a$ quickly.

\end{document}
